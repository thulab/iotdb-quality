\documentclass[cn,11pt,mode=simple,table]{elegantbook}
% 请使用XeLatex命令进行编译

\title{IoTDB-Quality用户文档}

\author{数据质量组}
\institute{清华大学软件学院}
\date{\today}
%\version{}
%\logo{logo-blue.png}
\cover{cover.jpg}

% 使用XeLatex编译时,item前面的圆点可能无法显示,下面的代码可以解决该问题
%\makeatletter
%\def\pgfutil@insertatbegincurrentpagefrombox#1{%
%  \edef\pgf@temp{\the\wd\pgfutil@abb}%
%  \global\setbox\pgfutil@abb\hbox{%
%    \unhbox\pgfutil@abb%
%    \hskip\dimexpr2in-2\hoffset-\pgf@temp\relax% changed
%    #1%
%    \hskip\dimexpr-2in-2\hoffset\relax% new
%  }%
%}
%\makeatother

% 本文档命令
\renewcommand{\today}{\number\year 年 \number\month 月 \number\day 日}
\newlength\tablewidth

\definecolor{AliceBlue}{RGB}{240,248,255}
\definecolor{LightGray}{RGB}{211,211,211}
\definecolor{tablelinegray}{RGB}{221,221,221}
\definecolor{tablerowgray}{RGB}{247,247,247}
\definecolor{tabletopgray}{RGB}{245,246,250}
\definecolor{airforceblue}{rgb}{0.36, 0.54, 0.66}

\lstset{
  breaklines=true,%自动换行
  columns=fixed,
  basicstyle=\footnotesize,
  keywordstyle=\color[RGB]{40,40,255},
  basewidth=0.45em,
  frameround=tttt,
  backgroundcolor=\color{AliceBlue}
}

\begin{document}

\maketitle
\tableofcontents
\mainmatter

\chapter{开始}
\input{output_zh/README.tex}
\input{output_zh/Comparison.tex}
\input{output_zh/QA.tex}

\chapter{数据画像}
\input{output_zh/Cov.tex}
\input{output_zh/Distinct.tex}
\input{output_zh/Histogram.tex}
\input{output_zh/Integral.tex}
\input{output_zh/Mean.tex}
\input{output_zh/Median.tex}
\input{output_zh/Mode.tex}
\input{output_zh/Pearson.tex}
\input{output_zh/Percentile.tex}
\input{output_zh/Sample.tex}
\input{output_zh/Skew.tex}
\input{output_zh/Spread.tex}
\input{output_zh/Stddev.tex}


\chapter{数据质量}
\input{output_zh/Completeness.tex}
\input{output_zh/Consistency.tex}
\input{output_zh/Timeliness.tex}
\input{output_zh/Validity.tex}

\chapter{数据修复}
\input{output_zh/Fill.tex}
\input{output_zh/ValueRepair.tex}
\input{output_zh/TimestampRepair.tex}

\chapter{数据匹配}
\input{output_zh/SeriesAlign.tex}
\input{output_zh/SeriesSimilarity.tex}
\input{output_zh/DTW.tex}

\chapter{异常检测}
\input{output_zh/Range.tex}
\input{output_zh/KSigma.tex}
\input{output_zh/LOF.tex}

\chapter{复杂事件处理}
\input{output_zh/EventMatching.tex}
\input{output_zh/MissingEventRecovery.tex}
\input{output_zh/EventNameRepair.tex}
\input{output_zh/EventTimeRepair.tex}
\input{output_zh/SEQ.tex}
\input{output_zh/AND.tex}

\end{document}
