\documentclass[11pt,color=green,table]{elegantbook}
% 请使用XeLatex命令进行编译

\title{User Manual of IoTDB-Quality}

\author{Data Quality Group}
\institute{School of Software, Tsinghua University}
\date{\today}
%\version{}
%\logo{logo-blue.png}
\cover{cover.jpg}

% 使用XeLatex编译时,item前面的圆点可能无法显示,下面的代码可以解决该问题
%\makeatletter
%\def\pgfutil@insertatbegincurrentpagefrombox#1{%
%  \edef\pgf@temp{\the\wd\pgfutil@abb}%
%  \global\setbox\pgfutil@abb\hbox{%
%    \unhbox\pgfutil@abb%
%    \hskip\dimexpr2in-2\hoffset-\pgf@temp\relax% changed
%    #1%
%    \hskip\dimexpr-2in-2\hoffset\relax% new
%  }%
%}
%\makeatother

% 本文档命令
\newlength\tablewidth
\definecolor{AliceBlue}{RGB}{240,248,255}
\definecolor{LightGray}{RGB}{211,211,211}
\definecolor{tablelinegray}{RGB}{221,221,221}
\definecolor{tablerowgray}{RGB}{247,247,247}
\definecolor{tabletopgray}{RGB}{245,246,250}
\definecolor{airforceblue}{rgb}{0.36, 0.54, 0.66}


\lstset{
  breaklines=true,%自动换行
  columns=fixed,
  basicstyle=\footnotesize,
  keywordstyle=\color[RGB]{40,40,255},
  basewidth=0.45em,
  frameround=tttt,
  backgroundcolor=\color{AliceBlue}
}

\begin{document}

\maketitle
\tableofcontents
\mainmatter

\chapter{Get Started}
\input{output_en/README.tex}
\input{output_en/Comparison.tex}
\input{output_en/QA.tex}

\chapter{Data Profiling}


\input{output_en/Distinct.tex}
\input{output_en/Histogram.tex}
\input{output_en/Integral.tex}
\input{output_en/Mad.tex}
\input{output_en/Median.tex}
\input{output_en/Mode.tex}
\input{output_en/Percentile.tex}
\input{output_en/Period.tex}
\input{output_en/QLB.tex}
\input{output_en/Resample.tex}
\input{output_en/Sample.tex}
\input{output_en/Segment.tex}

\input{output_en/Skew.tex}
\input{output_en/Spread.tex}
\input{output_en/Stddev.tex}
\input{output_en/TimeWeightedAvg.tex}

\chapter{Data Quality}
\input{output_en/Completeness.tex}
\input{output_en/Consistency.tex}
\input{output_en/Timeliness.tex}
\input{output_en/Validity.tex}

\chapter{Data Repairing}
\input{output_en/Fill.tex}
\input{output_en/TimestampRepair.tex}
\input{output_en/ValueRepair.tex}


\chapter{Data Matching}
\input{output_en/Cov.tex}
\input{output_en/CrossCorrelation.tex}
\input{output_en/DTW.tex}
\input{output_en/PatternSymmetric.tex}
\input{output_en/Pearson.tex}
\input{output_en/SelfCorrelation.tex}
\input{output_en/SeriesAlign.tex}
\input{output_en/SeriesSimilarity.tex}
\input{output_en/ValueAlign.tex}


\chapter{Anomaly Detection}
\input{output_en/ADWIN.tex}
\input{output_en/KSigma.tex}
\input{output_en/LOF.tex}
\input{output_en/Range.tex}
\input{output_en/TwoSidedFilter.tex}

\chapter{Frequency Domain}
\input{output_en/Conv.tex}
\input{output_en/Deconv.tex}
\input{output_en/FFT.tex}
\input{output_en/Filter.tex}


\chapter{Series Discovery}
\input{output_en/ConsecutiveSequences.tex}
\input{output_en/ConsecutiveWindows.tex}


\chapter{Complex Event Processing}
\input{output_en/AND.tex}
\input{output_en/EventMatching.tex}
\input{output_en/EventNameRepair.tex}
\input{output_en/EventTag.tex}
\input{output_en/EventTimeRepair.tex}
\input{output_en/MissingEventRecovery.tex}
\input{output_en/SEQ.tex}

\end{document}
